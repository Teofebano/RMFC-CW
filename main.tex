%
%                       This is a basic LaTeX Template
%                       for the Informatics Research Review

\documentclass[a4paper,11pt]{article}
% Add local fullpage and head macros
\usepackage{head,fullpage}     
% Add graphicx package with pdf flag (must use pdflatex)
\usepackage[pdftex]{graphicx}  
% Better support for URLs
\usepackage{url}
% Date formating
\usepackage{datetime}
% For Gantt chart
\usepackage{pgfgantt}
\usepackage{xcolor}
\usepackage[utf8]{inputenc}

\newdateformat{monthyeardate}{%
  \monthname[\THEMONTH] \THEYEAR}

\parindent=0pt          %  Switch off indent of paragraphs 
\parskip=5pt            %  Put 5pt between each paragraph  
\Urlmuskip=0mu plus 1mu %  Better line breaks for URLs


%                       This section generates a title page
%                       Edit only the following three lines
%                       providing your exam number, 
%                       the general field of study you are considering
%                       for your review, and name of IRR tutor

\newcommand{\examnumber}{B240710}
\newcommand{\field}{Using world-wide investor network to analyze the growth trajectories and survival rates of early-stage startups}
\newcommand{\tutor}{Felipe Costa Sperb}
\newcommand{\supervisor}{Valerio Restocchi}

\begin{document}
\begin{minipage}[b]{110mm}
        {\Huge\bf School of Informatics
        \vspace*{17mm}}
\end{minipage}
\hfill
\begin{minipage}[t]{40mm}               
        \makebox[40mm]{
        \includegraphics[width=40mm]{crest.png}}
\end{minipage}
\par\noindent
    % Centre Title, and name
\vspace*{2cm}
\begin{center}
        \Large\bf Informatics Project Proposal \\
        \Large\bf \field
\end{center}
\vspace*{1.5cm}
\begin{center}
        \bf \examnumber\\
        \monthyeardate\today
\end{center}
\vspace*{5mm}

%
%                       Insert your abstract HERE
%                       
\begin{abstract}
This research investigates how investor network characteristics and connectivity influence startup growth trajectories and survival rates. Utilizing network science, we construct a comprehensive model of investor-startup relationships and analyzing its centrality measurement, community detection, and their evolution over time. Our methodology integrates quantitative investment data interactions across temporal networks. Preliminary findings suggest that investor centrality and the structure of investment communities significantly affect startup outcomes. This study provides insights into effective investment strategies and supports startups in navigating the investment landscape, ultimately enhancing their likelihood of success.
\end{abstract}

\vspace*{1cm}

\vspace*{3cm}
Date: \today

\vfill
{\bf Tutor:} \tutor\\
{\bf Supervisor:} \supervisor
\newpage

%                                               Through page and setup 
%                                               fancy headings
\setcounter{page}{1}                            % Set page number to 1
\footruleheight{1pt}
\headruleheight{1pt}
\lfoot{\small School of Informatics}
\lhead{Informatics Research Review}
\rhead{- \thepage}
\cfoot{}
\rfoot{Date: \date{\today}}
%


\section{Motivation}
The startup ecosystem is a dynamic and complex arena where the flow of capital, innovation, and human resources intermingle to catalyze the growth of new enterprises. In this vibrant environment, the role of investor networks is paramount, not just as sources of funding but as crucial conduits for guidance, expertise, and industry connections. This research delves into the structural nuances of these networks, examining how their configuration influences the success trajectories of startups.

Investors (encompassing angels, venture capitalists, and corporate investors) naturally create a web of relationships that can significantly impact a startup’s access to resources, market entry, and competitive positioning. Unfortunately, despite the abundance of data in this particular network, less is understood about how the network aspects of these relationships—such as centrality and community structure—affect startup outcomes. This gap presents a critical area for exploration, particularly given the increasing reliance on data-driven decision-making in venture capital.

The scope of this project is to construct and analyze a detailed network model of these investor-startup relationships, revealing underlying patterns and behaviors that might contribute to startups' success. This information can thereby informing investment strategies and potentially guiding policy-making in economic hubs. This approach not only bridges an important research gap but also enhances the strategic frameworks within which venture capitalists operate.

This research is motivated by the potential to transform the conventional methods of startup evaluation and to offer a more nuanced understanding of how structural network properties can be leveraged to forecast and enhance startup viability and growth. By engaging with this topic, the study aims to contribute significantly to the academic discourse and provide actionable insights that could reshape investment strategies in the entrepreneurial landscape.

\subsection{Problem Statement}
As stated before, despite the critical role of investor networks in the startup ecosystem, there remains a significant gap in understanding how the specific structures and characteristics of these networks influence startup outcomes. This research aims to address the problem: How do the connectivity and characteristics of investor networks affect the growth trajectories and survival rates of startups?

The startup landscape is rich with data on investment amounts and funding rounds, yet little is quantitatively known about the impact of the structural properties of investor networks, such as centrality of investors, the configuration of investment communities, and their evolution over time. Most current models focus predominantly on financial metrics or founder backgrounds, overlooking the broader network dynamics that can significantly influence startup success. This study seeks to fill this gap by focusing on measurable network metrics mentioned above as predictors of startup performance. By limiting the investigation to these network characteristics, the research aims to provide a more nuanced understanding of how investor relationships within these networks directly correlate with startup growth and sustainability, thereby offering a foundational tool for strategic decision-making in the investment community.

\subsection{Research Hypothesis and Objectives}
This research investigates the hypothesis that the connectivity and characteristics of investor networks, specifically the centrality measures of investors and the structure of investment communities, significantly influence the growth trajectories and survival rates of startups. We hypothesize that startups linked to highly central investors within cohesive investment communities are likely to demonstrate higher growth rates and improved survival prospects.

There are several objectives in this research. 
\begin{itemize}
    \item Construct a comprehensive network model of investor-startup relationships that incorporates quantitative investment data
    \item Analyze the centrality of investors within the network to assess how these measures correlate with the success metrics of funded startups
    \item Identify and analyze the structure of communities within the investor network to determine their impact on startup outcomes
    \item Examine the evolution of the investor-startup network over time, capturing dynamic changes and their effects on startup growth and survival
    \item Examine the evolution of the investor-startup network over time, capturing dynamic changes and their effects on startup growth and survival
\end{itemize}

The rationale behind this study is driven by the recognized importance of network effects in the startup ecosystem, where relationships dictate access to resources, knowledge, and opportunities. By understanding the structural attributes of investor networks, this research aims to provide critical insights into the mechanisms that foster startup success, offering a competitive edge to both investors and startups.

However, the study will not explore the qualitative aspects of investor-startup relationships, such as the quality of mentorship or the level of investor involvement, beyond their quantifiable impact through network structures. Furthermore, the focus will be primarily on quantifiable network metrics and their direct correlations, rather than on external factors like market conditions or technological innovations that could also significantly impact startup success. This approach ensures a focused investigation into the network dynamics while acknowledging the broader ecosystem's complexity and multifaceted nature.

\subsection{Timeliness and Novelty}
In today’s highly interconnected business environment, network effects have become more pronounced, especially in the startup ecosystem. Understanding how the structure of these networks affects startup success is increasingly critical. The proposed research addresses this by examining how investor networks, a relatively under-explored aspect, impact startup outcomes. This approach is particularly timely given the growing importance of social capital and network leverage in startup viability and success.

Furthermore, while there is extensive research on the financial aspects of startup success and the individual characteristics of entrepreneurs and their teams, less attention has been paid to the influence of investor networks' structural properties. This research fills this gap, offering new insights that could be critical for both academic understanding and practical applications in business strategy and economic policy.

On the other hand, the venture capital and broader investment community are increasingly seeking data-driven approaches to improve investment decisions and outcomes. This research offers a novel approach by focusing on structural network metrics, providing a new angle from which to assess potential investments. This could revolutionize how investors identify promising startups and allocate resources, aligning well with current trends towards more strategic, informed investment practices.

Additionally, it is important for startups to understand how to position themselves within their investor network can be as crucial as securing funding. This research is novel in that it not only helps startups understand their position within these networks but also provides them with knowledge on how to strategically enhance their network position to maximize their success potential.

In summary, the proposed research is timely due to the current economic and technological landscape and novel because of its focus on an under-explored yet crucial aspect of startup success. This combination makes it particularly relevant and valuable to both the academic community and the broader industry, including startups and investors alike.

\subsection{Significance (WIP)}

The proposal should demonstrate the originality of your intended research. You should therefore explain why your research is important (for example, by explaining how your research builds on and adds to the current state of knowledge in the field or by setting out reasons why it is timely to research your proposed topic) and providing details of any immediate applications, including further research that might be done to build on your findings.

\subsection{Feasibility (WIP)}

Comment on the feasibility of the research plans given its limited time frame and resources. Outline your plans for a feasibility study before starting e.g.\ major implementation work.

\subsection{Beneficiaries (WIP)}

Describe how the research will benefit other researchers in the field and in related disciplines. What will be done to ensure that they can benefit? 


\section{Background and Related Work (WIP)}

Demonstrate a knowledge and understanding of past and current work in the subject area, including relevant references like this \cite{template}.


\section{Programme and Methodology (WIP)}

\begin{itemize}
    \item Detail the methodology to be used in pursuit of the research and justify this choice.
    \item Describe your contributions and novelty and where you
    will go beyond the state-of-the-art (new methods, new tools,
    new data, new insights, new proofs,...)
    \item Describe the programme of work, indicating the research to be undertaken and the milestones that can be used to measure its progress.
    \item Where suitable define work packages and define the dependences
    between these work packages. WPs and their dependences should be
    shown in the Gantt chart in the research plan.
    \item Explain how the project will be managed.
    \item State the limitations of your research.
\end{itemize}

\subsection{Risk Assessment (WIP)}

\subsection{Ethics (WIP)}


\section{Evaluation}
\subsection{Data Collection}
The primary method of data collection for this research involves utilizing the Crunchbase API, a comprehensive source of information about startups, including details about their investors, funding rounds, and other relevant data. This approach allows for systematic and scalable data extraction, ensuring that we capture a broad and detailed dataset encompassing various aspects of the startup ecosystem. Specifically, we will extract data on:
\begin{itemize}
    \item Startups: Names, industry sectors, dates of establishment, funding histories, and current status (open deals, acquisitions, IPOs, etc.).
    \item Investors: Types (e.g., angel, venture capital), investment patterns, and network links to other investors and startups.
    \item Investment Transactions: Amounts, dates, and the sequence of funding rounds to trace the flow of capital within the network.
\end{itemize}

This extensive data collection will enable a comprehensive analysis of the investor-startup network, providing a robust foundation for the subsequent network construction and analysis.

\subsection{Result Interpretation}
To analyze and interpret the results, the research will focus on startups classified as "open deals"—those that have not yet received funding, been acquired, or been listed on the stock exchange. By examining these startups, the study aims to assess the predictive power of early-stage network metrics, specifically centrality and community structure, on long-term economic performance.

Generally, it will follow this step-by-step : 
\begin{itemize}
    \item Using the collected data, we will construct a network where nodes represent startups and investors, and edges represent investment relationships. The network will be analyzed over time to capture the dynamics of investor influence and startup development
    \item Calculate centrality measures for each startup in the network. This analysis will identify startups that, at an early stage, are centrally positioned within the investor network, hypothesizing that these startups are more likely to succeed
    \item Apply algorithms to detect communities within the network, assuming that startups within highly interconnected investor communities may have better access to resources and knowledge, contributing to their success
    \item Analyze startups' success based on whether, within a 7-year window from founding, startups achieve significant milestones such as acquiring other firms, being acquired, or completing an IPO
    \item Compare the early predictions based on network centrality and community detection with actual outcomes to evaluate the accuracy and predictive power of these network metrics
\end{itemize}

By integrating these methods, the study aims to substantiate the hypothesis that structural properties within investor networks can serve as early indicators of a startup's long-term economic performance. This analysis will not only contribute to academic knowledge but also provide practical insights for investors and policymakers to refine their strategies and interventions in the startup ecosystem.

\section{Expected Outcomes (WIP)}

Conclude your research proposal by addressing your predicted outcomes. What are you hoping to prove/disprove? Indicate how you envisage your research will contribute to debates and discussions in your particular subject area:

\begin{itemize}
    \item How will your research make an original contribution to knowledge?
    \item How might it fill gaps in existing work? 
    \item How might it extend understanding of particular topics?
\end{itemize}


\section{Research Plan, Milestones and Deliverables (WIP)}

\definecolor{barblue}{RGB}{153,204,254}
\definecolor{groupblue}{RGB}{51,102,254}
\definecolor{linkred}{RGB}{165,0,33} 

\begin{figure}[htbp]
\begin{ganttchart}[
    y unit title=0.4cm,
    y unit chart=0.5cm,
    vgrid,hgrid,
    x unit=1.55mm,
    time slot format=isodate,
    title/.append style={draw=none, fill=barblue},
    title label font=\sffamily\bfseries\color{white},
    title label node/.append style={below=-1.6ex},
    title left shift=.05,
    title right shift=-.05,
    title height=1,
    bar/.append style={draw=none, fill=groupblue},
    bar height=.6,
    bar label font=\normalsize\color{black!50},
    group right shift=0,
    group top shift=.6,
    group height=.3,
    group peaks height=.2,
    bar incomplete/.append style={fill=green}
   ]{2018-06-01}{2018-08-16}
   \gantttitlecalendar{month=name}\\
   \ganttbar[
    progress=100,
    bar progress label font=\small\color{barblue},
    bar progress label node/.append style={right=4pt},
    bar label font=\normalsize\color{barblue},
    name=pp
   ]{Background Reading}{2018-06-01}{2018-06-14} \\
\ganttset{progress label text={}, link/.style={black, -to}}
\ganttgroup{Work Package 1}{2018-06-14}{2018-06-30} \\
\ganttbar[progress=4, name=T1A]{Task A}{2018-06-14}{2018-06-21} \\
\ganttlinkedbar[progress=0]{Task B}{2018-06-21}{2018-06-30} \\
\ganttgroup{Work Package 2}{2018-07-01}{2018-07-14} \\
\ganttbar[progress=15, name=T2A]{Task A}{2018-07-01}{2018-07-07} \\
\ganttlinkedbar[progress=0]{Task B}{2018-07-07}{2018-07-14} \\
\ganttgroup{Dissertation}{2018-07-14}{2018-08-16} \\
  \ganttbar[progress=0]{Task A}{2018-07-14}{2018-08-16}
  \ganttset{link/.style={green}}
  \ganttlink[link mid=.4]{pp}{T1A}
  \ganttlink[link mid=.159]{pp}{T2A}
\end{ganttchart}
\caption{Gantt Chart of the activities defined for this project.}
\label{fig:gantt}
\end{figure}

\begin{table}[htbp]
    \begin{center}
        \begin{tabular}{|c|c|l|}
        \hline
        \textbf{Milestone} & \textbf{Week} & \textbf{Description} \\
        \hline
        $M_1$ & 2 & Feasibility study completed \\
        $M_2$ & 5 & First prototype implementation completed \\
        $M_3$ & 7 & Evaluation completed \\
        $M_4$ & 10 & Submission of dissertation \\
        \hline
        \end{tabular} 
    \end{center}
    \caption{Milestones defined in this project.}
    \label{fig:milestones}
\end{table}

\begin{table}[htbp]
    \begin{center}
        \begin{tabular}{|c|c|l|}
        \hline
        \textbf{Deliverable} & \textbf{Week} & \textbf{Description} \\
        \hline
        $D_1$ & 6 & Software tool for \dots\\
        $D_2$ & 8 & Evaluation report on \dots\\
        $D_3$ & 10 & Dissertation \\
        \hline
        \end{tabular} 
    \end{center}
    \caption{List of deliverables defined in this project.}
    \label{fig:deliverables}
\end{table}


%                Now build the reference list
\bibliographystyle{unsrt}   % The reference style
%                This is plain and unsorted, so in the order
%                they appear in the document.

{\small
\bibliography{main}       % bib file(s).
}
\end{document}

